\chapter{What happened to them at Surinam and how Candide got acquainted with Martin}
\vspace{-1cm}
\lettrine[lraise=0.1,nindent=0em,slope=-.5em]{O}{ur} travellers spent the first day very agreeably. They were delighted with possessing more treasure than all Asia, Europe, and Africa could scrape together. Candide, in his raptures, cut Cunegonde's name on the trees. The second day two of their sheep plunged into a morass, where they and their burdens were lost; two more died of fatigue a few days after; seven or eight perished with hunger in a desert; and others subsequently fell down precipices. At length, after travelling a hundred days, only two sheep remained. Said Candide to Cacambo:

``My friend, you see how perishable are the riches of this world; there is nothing solid but virtue, and the happiness of seeing Cunegonde once more.''

``I grant all you say,'' said Cacambo, ``but we have still two sheep remaining, with more treasure than the King of Spain will ever have; and I see a town which I take to be Surinam, belonging to the Dutch. We are at the end of all our troubles, and at the beginning of happiness.''

As they drew near the town, they saw a negro stretched upon the ground, with only one moiety of his clothes, that is, of his blue linen drawers; the poor man had lost his left leg and his right hand.

``Good God!'' said Candide in Dutch, ``what art thou doing there, friend, in that shocking condition?''

``I am waiting for my master, Mynheer Vanderdendur, the famous merchant,'' answered the negro.

``Was it Mynheer Vanderdendur,'' said Candide, ``that treated thee thus?''

``Yes, sir,'' said the negro, ``it is the custom. They give us a pair of linen drawers for our whole garment twice a year. When we work at the sugar-canes, and the mill snatches hold of a finger, they cut off the hand; and when we attempt to run away, they cut off the leg; both cases have happened to me. This is the price at which you eat sugar in Europe. Yet when my mother sold me for ten patagons\footnote{Spanish half-crowns.} on the coast of Guinea, she said to me: 'My dear child, bless our fetiches, adore them for ever; they will make thee live happily; thou hast the honour of being the slave of our lords, the whites, which is making the fortune of thy father and mother.' Alas! I know not whether I have made their fortunes; this I know, that they have not made mine. Dogs, monkeys, and parrots are a thousand times less wretched than I. The Dutch fetiches, who have converted me, declare every Sunday that we are all of us children of Adam--blacks as well as whites. I am not a genealogist, but if these preachers tell truth, we are all second cousins. Now, you must agree, that it is impossible to treat one's relations in a more barbarous manner.''

``Oh, Pangloss!'' cried Candide, ``thou hadst not guessed at this abomination; it is the end. I must at last renounce thy optimism.''

``What is this optimism?'' said Cacambo.

``Alas!'' said Candide, ``it is the madness of maintaining that everything is right when it is wrong.''

Looking at the negro, he shed tears, and weeping, he entered Surinam.

The first thing they inquired after was whether there was a vessel in the harbour which could be sent to Buenos Ayres. The person to whom they applied was a Spanish sea-captain, who offered to agree with them upon reasonable terms. He appointed to meet them at a public-house, whither Candide and the faithful Cacambo went with their two sheep, and awaited his coming.

Candide, who had his heart upon his lips, told the Spaniard all his adventures, and avowed that he intended to elope with Miss Cunegonde.

``Then I will take good care not to carry you to Buenos Ayres,'' said the seaman. ``I should be hanged, and so would you. The fair Cunegonde is my lord's favourite mistress.''

This was a thunderclap for Candide: he wept for a long while. At last he drew Cacambo aside.

``Here, my dear friend,'' said he to him, ``this thou must do. We have, each of us in his pocket, five or six millions in diamonds; you are more clever than I; you must go and bring Miss Cunegonde from Buenos Ayres. If the Governor makes any difficulty, give him a million; if he will not relinquish her, give him two; as you have not killed an Inquisitor, they will have no suspicion of you; I'll get another ship, and go and wait for you at Venice; that's a free country, where there is no danger either from Bulgarians, Abares, Jews, or Inquisitors.''

Cacambo applauded this wise resolution. He despaired at parting from so good a master, who had become his intimate friend; but the pleasure of serving him prevailed over the pain of leaving him. They embraced with tears; Candide charged him not to forget the good old woman. Cacambo set out that very same day. This Cacambo was a very honest fellow.

Candide stayed some time longer in Surinam, waiting for another captain to carry him and the two remaining sheep to Italy. After he had hired domestics, and purchased everything necessary for a long voyage, Mynheer Vanderdendur, captain of a large vessel, came and offered his services.

``How much will you charge,'' said he to this man, ``to carry me straight to Venice--me, my servants, my baggage, and these two sheep?''

The skipper asked ten thousand piastres. Candide did not hesitate.

``Oh! oh!'' said the prudent Vanderdendur to himself, ``this stranger gives ten thousand piastres unhesitatingly! He must be very rich.''

Returning a little while after, he let him know that upon second consideration, he could not undertake the voyage for less than twenty thousand piastres.

``Well, you shall have them,'' said Candide.

``Ay!'' said the skipper to himself, ``this man agrees to pay twenty thousand piastres with as much ease as ten.''

He went back to him again, and declared that he could not carry him to Venice for less than thirty thousand piastres.

``Then you shall have thirty thousand,'' replied Candide.

``Oh! oh!'' said the Dutch skipper once more to himself, ``thirty thousand piastres are a trifle to this man; surely these sheep must be laden with an immense treasure; let us say no more about it. First of all, let him pay down the thirty thousand piastres; then we shall see.''

Candide sold two small diamonds, the least of which was worth more than what the skipper asked for his freight. He paid him in advance. The two sheep were put on board. Candide followed in a little boat to join the vessel in the roads. The skipper seized his opportunity, set sail, and put out to sea, the wind favouring him. Candide, dismayed and stupefied, soon lost sight of the vessel.

``Alas!'' said he, ``this is a trick worthy of the old world!''

He put back, overwhelmed with sorrow, for indeed he had lost sufficient to make the fortune of twenty monarchs. He waited upon the Dutch magistrate, and in his distress he knocked over loudly at the door. He entered and told his adventure, raising his voice with unnecessary vehemence. The magistrate began by fining him ten thousand piastres for making a noise; then he listened patiently, promised to examine into his affair at the skipper's return, and ordered him to pay ten thousand piastres for the expense of the hearing.

This drove Candide to despair; he had, indeed, endured misfortunes a thousand times worse; the coolness of the magistrate and of the skipper who had robbed him, roused his choler and flung him into a deep melancholy. The villainy of mankind presented itself before his imagination in all its deformity, and his mind was filled with gloomy ideas. At length hearing that a French vessel was ready to set sail for Bordeaux, as he had no sheep laden with diamonds to take along with him he hired a cabin at the usual price. He made it known in the town that he would pay the passage and board and give two thousand piastres to any honest man who would make the voyage with him, upon condition that this man was the most dissatisfied with his state, and the most unfortunate in the whole province.

Such a crowd of candidates presented themselves that a fleet of ships could hardly have held them. Candide being desirous of selecting from among the best, marked out about one-twentieth of them who seemed to be sociable men, and who all pretended to merit his preference. He assembled them at his inn, and gave them a supper on condition that each took an oath to relate his history faithfully, promising to choose him who appeared to be most justly discontented with his state, and to bestow some presents upon the rest.

They sat until four o'clock in the morning. Candide, in listening to all their adventures, was reminded of what the old woman had said to him in their voyage to Buenos Ayres, and of her wager that there was not a person on board the ship but had met with very great misfortunes. He dreamed of Pangloss at every adventure told to him.

``This Pangloss,'' said he, ``would be puzzled to demonstrate his system. I wish that he were here. Certainly, if all things are good, it is in El Dorado and not in the rest of the world.''

At length he made choice of a poor man of letters, who had worked ten years for the booksellers of Amsterdam. He judged that there was not in the whole world a trade which could disgust one more.

This philosopher was an honest man; but he had been robbed by his wife, beaten by his son, and abandoned by his daughter who got a Portuguese to run away with her. He had just been deprived of a small employment, on which he subsisted; and he was persecuted by the preachers of Surinam, who took him for a Socinian. We must allow that the others were at least as wretched as he; but Candide hoped that the philosopher would entertain him during the voyage. All the other candidates complained that Candide had done them great injustice; but he appeased them by giving one hundred piastres to each.


