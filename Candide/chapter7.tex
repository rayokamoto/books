\vspace{1cm}
\begingroup
\let\clearpage\relax
\chapter{How the Old Woman took care of Candide, and how he found the Object he loved}
\endgroup
\vspace{-1cm}
\lettrine[lraise=0.1,nindent=0em,slope=-.5em]{C}{andide} did not take courage, but followed the old woman to a decayed house, where she gave him a pot of pomatum to anoint his sores, showed him a very neat little bed, with a suit of clothes hanging up, and left him something to eat and drink.

``Eat, drink, sleep,'' said she, ``and may our lady of Atocha,\footnote{''This Notre-Dame is of wood; every year she weeps on the day of her \textit{fete}, and the people weep also. One day the preacher, seeing a carpenter with dry eyes, asked him how it was that he did not dissolve in tears when the Holy Virgin wept. 'Ah, my reverend father,' replied he, 'it is I who refastened her in her niche yesterday. I drove three great nails through her behind; it is then she would have wept if she had been able.'``--Voltaire, \textit{Melanges}.} the great St. Anthony of Padua, and the great St. James of Compostella, receive you under their protection. I shall be back to-morrow.''

Candide, amazed at all he had suffered and still more with the charity of the old woman, wished to kiss her hand.

``It is not my hand you must kiss,'' said the old woman; ``I shall be back to-morrow. Anoint yourself with the pomatum, eat and sleep.''

Candide, notwithstanding so many disasters, ate and slept. The next morning the old woman brought him his breakfast, looked at his back, and rubbed it herself with another ointment: in like manner she brought him his dinner; and at night she returned with his supper. The day following she went through the very same ceremonies.

``Who are you?'' said Candide; ``who has inspired you with so much goodness? What return can I make you?''

The good woman made no answer; she returned in the evening, but brought no supper.

``Come with me,'' she said, ``and say nothing.''

She took him by the arm, and walked with him about a quarter of a mile into the country; they arrived at a lonely house, surrounded with gardens and canals. The old woman knocked at a little door, it opened, she led Candide up a private staircase into a small apartment richly furnished. She left him on a brocaded sofa, shut the door and went away. Candide thought himself in a dream; indeed, that he had been dreaming unluckily all his life, and that the present moment was the only agreeable part of it all.

The old woman returned very soon, supporting with difficulty a trembling woman of a majestic figure, brilliant with jewels, and covered with a veil.

``Take off that veil,'' said the old woman to Candide.

The young man approaches, he raises the veil with a timid hand. Oh! what a moment! what surprise! he believes he beholds Miss Cunegonde? he really sees her! it is herself! His strength fails him, he cannot utter a word, but drops at her feet. Cunegonde falls upon the sofa. The old woman supplies a smelling bottle; they come to themselves and recover their speech. As they began with broken accents, with questions and answers interchangeably interrupted with sighs, with tears, and cries. The old woman desired they would make less noise and then she left them to themselves.

``What, is it you?'' said Candide, ``you live? I find you again in Portugal? then you have not been ravished? then they did not rip open your belly as Doctor Pangloss informed me?''

``Yes, they did,'' said the beautiful Cunegonde; ``but those two accidents are not always mortal.''

``But were your father and mother killed?''

``It is but too true,'' answered Cunegonde, in tears.

``And your brother?''

``My brother also was killed.''

``And why are you in Portugal? and how did you know of my being here? and by what strange adventure did you contrive to bring me to this house?''

``I will tell you all that,'' replied the lady, ``but first of all let me know your history, since the innocent kiss you gave me and the kicks which you received.''

Candide respectfully obeyed her, and though he was still in a surprise, though his voice was feeble and trembling, though his back still pained him, yet he gave her a most ingenuous account of everything that had befallen him since the moment of their separation. Cunegonde lifted up her eyes to heaven; shed tears upon hearing of the death of the good Anabaptist and of Pangloss; after which she spoke as follows to Candide, who did not lose a word and devoured her with his eyes.

