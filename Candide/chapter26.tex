\chapter{Of a supper which Candide and Martin took with Six Strangers, and who they were}
\lettrine[lraise=0.1,nindent=0em,slope=-.5em]{O}{ne} evening that Candide and Martin were going to sit down to supper with some foreigners who lodged in the same inn, a man whose complexion was as black as soot, came behind Candide, and taking him by the arm, said:

``Get yourself ready to go along with us; do not fail.''

Upon this he turned round and saw--Cacambo! Nothing but the sight of Cunegonde could have astonished and delighted him more. He was on the point of going mad with joy. He embraced his dear friend.

``Cunegonde is here, without doubt; where is she? Take me to her that I may die of joy in her company.''

``Cunegonde is not here,'' said Cacambo, ``she is at Constantinople.''

``Oh, heavens! at Constantinople! But were she in China I would fly thither; let us be off.''

``We shall set out after supper,'' replied Cacambo. ``I can tell you nothing more; I am a slave, my master awaits me, I must serve him at table; speak not a word, eat, and then get ready.''

Candide, distracted between joy and grief, delighted at seeing his faithful agent again, astonished at finding him a slave, filled with the fresh hope of recovering his mistress, his heart palpitating, his understanding confused, sat down to table with Martin, who saw all these scenes quite unconcerned, and with six strangers who had come to spend the Carnival at Venice.

Cacambo waited at table upon one of the strangers; towards the end of the entertainment he drew near his master, and whispered in his ear:

``Sire, your Majesty may start when you please, the vessel is ready.''

On saying these words he went out. The company in great surprise looked at one another without speaking a word, when another domestic approached his master and said to him:

``Sire, your Majesty's chaise is at Padua, and the boat is ready.''

The master gave a nod and the servant went away. The company all stared at one another again, and their surprise redoubled. A third valet came up to a third stranger, saying:

``Sire, believe me, your Majesty ought not to stay here any longer. I am going to get everything ready.''

And immediately he disappeared. Candide and Martin did not doubt that this was a masquerade of the Carnival. Then a fourth domestic said to a fourth master:

``Your Majesty may depart when you please.''

Saying this he went away like the rest. The fifth valet said the same thing to the fifth master. But the sixth valet spoke differently to the sixth stranger, who sat near Candide. He said to him:

``Faith, Sire, they will no longer give credit to your Majesty nor to me, and we may perhaps both of us be put in jail this very night. Therefore I will take care of myself. Adieu.''

The servants being all gone, the six strangers, with Candide and Martin, remained in a profound silence. At length Candide broke it.

``Gentlemen,'' said he, ``this is a very good joke indeed, but why should you all be kings? For me I own that neither Martin nor I is a king.''

Cacambo's master then gravely answered in Italian:

``I am not at all joking. My name is Achmet III. I was Grand Sultan many years. I dethroned my brother; my nephew dethroned me, my viziers were beheaded, and I am condemned to end my days in the old Seraglio. My nephew, the great Sultan Mahmoud, permits me to travel sometimes for my health, and I am come to spend the Carnival at Venice.''

A young man who sat next to Achmet, spoke then as follows:

``My name is Ivan. I was once Emperor of all the Russias, but was dethroned in my cradle. My parents were confined in prison and I was educated there; yet I am sometimes allowed to travel in company with persons who act as guards; and I am come to spend the Carnival at Venice.''

The third said:

``I am Charles Edward, King of England; my father has resigned all his legal rights to me. I have fought in defence of them; and above eight hundred of my adherents have been hanged, drawn, and quartered. I have been confined in prison; I am going to Rome, to pay a visit to the King, my father, who was dethroned as well as myself and my grandfather, and I am come to spend the Carnival at Venice.''

The fourth spoke thus in his turn:

``I am the King of Poland; the fortune of war has stripped me of my hereditary dominions; my father underwent the same vicissitudes; I resign myself to Providence in the same manner as Sultan Achmet, the Emperor Ivan, and King Charles Edward, whom God long preserve; and I am come to the Carnival at Venice.''

The fifth said:

``I am King of Poland also; I have been twice dethroned; but Providence has given me another country, where I have done more good than all the Sarmatian kings were ever capable of doing on the banks of the Vistula; I resign myself likewise to Providence, and am come to pass the Carnival at Venice.''

It was now the sixth monarch's turn to speak:

``Gentlemen,'' said he, ``I am not so great a prince as any of you; however, I am a king. I am Theodore, elected King of Corsica; I had the title of Majesty, and now I am scarcely treated as a gentleman. I have coined money, and now am not worth a farthing; I have had two secretaries of state, and now I have scarce a valet; I have seen myself on a throne, and I have seen myself upon straw in a common jail in London. I am afraid that I shall meet with the same treatment here though, like your majesties, I am come to see the Carnival at Venice.''

The other five kings listened to this speech with generous compassion. Each of them gave twenty sequins to King Theodore to buy him clothes and linen; and Candide made him a present of a diamond worth two thousand sequins.

``Who can this private person be,'' said the five kings to one another, ``who is able to give, and really has given, a hundred times as much as any of us?''

Just as they rose from table, in came four Serene Highnesses, who had also been stripped of their territories by the fortune of war, and were come to spend the Carnival at Venice. But Candide paid no regard to these newcomers, his thoughts were entirely employed on his voyage to Constantinople, in search of his beloved Cunegonde.


