\chapter{Adventures of the Two Travellers, with Two Girls, Two Monkeys, and the Savages called Oreillons}
\lettrine[lraise=0.1,nindent=0em,slope=-.5em]{C}{andide} and his valet had got beyond the barrier, before it was known in the camp that the German Jesuit was dead. The wary Cacambo had taken care to fill his wallet with bread, chocolate, bacon, fruit, and a few bottles of wine. With their Andalusian horses they penetrated into an unknown country, where they perceived no beaten track. At length they came to a beautiful meadow intersected with purling rills. Here our two adventurers fed their horses. Cacambo proposed to his master to take some food, and he set him an example.

``How can you ask me to eat ham,'' said Candide, ``after killing the Baron's son, and being doomed never more to see the beautiful Cunegonde? What will it avail me to spin out my wretched days and drag them far from her in remorse and despair? And what will the \textit{Journal of Trevoux}\footnote{By the \textit{Journal of Trevoux} Voltaire meant a critical periodical printed by the Jesuits at Trevoux under the title of \textit{Memoires pour servir a l'Historie des Sciences et des Beaux-Arts}. It existed from 1701 until 1767, during which period its title underwent many changes.} say?''

While he was thus lamenting his fate, he went on eating. The sun went down. The two wanderers heard some little cries which seemed to be uttered by women. They did not know whether they were cries of pain or joy; but they started up precipitately with that inquietude and alarm which every little thing inspires in an unknown country. The noise was made by two naked girls, who tripped along the mead, while two monkeys were pursuing them and biting their buttocks. Candide was moved with pity; he had learned to fire a gun in the Bulgarian service, and he was so clever at it, that he could hit a filbert in a hedge without touching a leaf of the tree. He took up his double-barrelled Spanish fusil, let it off, and killed the two monkeys.

``God be praised! My dear Cacambo, I have rescued those two poor creatures from a most perilous situation. If I have committed a sin in killing an Inquisitor and a Jesuit, I have made ample amends by saving the lives of these girls. Perhaps they are young ladies of family; and this adventure may procure us great advantages in this country.''

He was continuing, but stopped short when he saw the two girls tenderly embracing the monkeys, bathing their bodies in tears, and rending the air with the most dismal lamentations.

``Little did I expect to see such good-nature,'' said he at length to Cacambo; who made answer:

``Master, you have done a fine thing now; you have slain the sweethearts of those two young ladies.''

``The sweethearts! Is it possible? You are jesting, Cacambo, I can never believe it!''

``Dear master,'' replied Cacambo; ``you are surprised at everything. Why should you think it so strange that in some countries there are monkeys which insinuate themselves into the good graces of the ladies; they are a fourth part human, as I am a fourth part Spaniard.''

``Alas!'' replied Candide, ``I remember to have heard Master Pangloss say, that formerly such accidents used to happen; that these mixtures were productive of Centaurs, Fauns, and Satyrs; and that many of the ancients had seen such monsters, but I looked upon the whole as fabulous.''

``You ought now to be convinced,'' said Cacambo, ``that it is the truth, and you see what use is made of those creatures, by persons that have not had a proper education; all I fear is that those ladies will play us some ugly trick.''

These sound reflections induced Candide to leave the meadow and to plunge into a wood. He supped there with Cacambo; and after cursing the Portuguese inquisitor, the Governor of Buenos Ayres, and the Baron, they fell asleep on moss. On awaking they felt that they could not move; for during the night the Oreillons, who inhabited that country, and to whom the ladies had denounced them, had bound them with cords made of the bark of trees. They were encompassed by fifty naked Oreillons, armed with bows and arrows, with clubs and flint hatchets. Some were making a large cauldron boil, others were preparing spits, and all cried:

``A Jesuit! a Jesuit! we shall be revenged, we shall have excellent cheer, let us eat the Jesuit, let us eat him up!''

``I told you, my dear master,'' cried Cacambo sadly, ``that those two girls would play us some ugly trick.''

Candide seeing the cauldron and the spits, cried:

``We are certainly going to be either roasted or boiled. Ah! what would Master Pangloss say, were he to see how pure nature is formed? Everything is right, may be, but I declare it is very hard to have lost Miss Cunegonde and to be put upon a spit by Oreillons.''

Cacambo never lost his head.

``Do not despair,'' said he to the disconsolate Candide, ``I understand a little of the jargon of these people, I will speak to them.''

``Be sure,'' said Candide, ``to represent to them how frightfully inhuman it is to cook men, and how very un-Christian.''

``Gentlemen,'' said Cacambo, ``you reckon you are to-day going to feast upon a Jesuit. It is all very well, nothing is more unjust than thus to treat your enemies. Indeed, the law of nature teaches us to kill our neighbour, and such is the practice all over the world. If we do not accustom ourselves to eating them, it is because we have better fare. But you have not the same resources as we; certainly it is much better to devour your enemies than to resign to the crows and rooks the fruits of your victory. But, gentlemen, surely you would not choose to eat your friends. You believe that you are going to spit a Jesuit, and he is your defender. It is the enemy of your enemies that you are going to roast. As for myself, I was born in your country; this gentleman is my master, and, far from being a Jesuit, he has just killed one, whose spoils he wears; and thence comes your mistake. To convince you of the truth of what I say, take his habit and carry it to the first barrier of the Jesuit kingdom, and inform yourselves whether my master did not kill a Jesuit officer. It will not take you long, and you can always eat us if you find that I have lied to you. But I have told you the truth. You are too well acquainted with the principles of public law, humanity, and justice not to pardon us.''

The Oreillons found this speech very reasonable. They deputed two of their principal people with all expedition to inquire into the truth of the matter; these executed their commission like men of sense, and soon returned with good news. The Oreillons untied their prisoners, showed them all sorts of civilities, offered them girls, gave them refreshment, and reconducted them to the confines of their territories, proclaiming with great joy:

``He is no Jesuit! He is no Jesuit!''

Candide could not help being surprised at the cause of his deliverance.

``What people!'' said he; ``what men! what manners! If I had not been so lucky as to run Miss Cunegonde's brother through the body, I should have been devoured without redemption. But, after all, pure nature is good, since these people, instead of feasting upon my flesh, have shown me a thousand civilities, when then I was not a Jesuit.''

