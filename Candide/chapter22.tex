\vspace{1cm}
\begingroup
\let\clearpage\relax
\chapter{What happened in France to Candide and Martin}
\endgroup
\vspace{-1cm}
\lettrine[lraise=0.1,nindent=0em,slope=-.5em]{C}{andide} stayed in Bordeaux no longer than was necessary for the selling of a few of the pebbles of El Dorado, and for hiring a good chaise to hold two passengers; for he could not travel without his Philosopher Martin. He was only vexed at parting with his sheep, which he left to the Bordeaux Academy of Sciences, who set as a subject for that year's prize, ``to find why this sheep's wool was red;'' and the prize was awarded to a learned man of the North, who demonstrated by A plus B minus C divided by Z, that the sheep must be red, and die of the rot.

Meanwhile, all the travellers whom Candide met in the inns along his route, said to him, ``We go to Paris.'' This general eagerness at length gave him, too, a desire to see this capital; and it was not so very great a \textit{detour} from the road to Venice.

He entered Paris by the suburb of St. Marceau, and fancied that he was in the dirtiest village of Westphalia.

Scarcely was Candide arrived at his inn, than he found himself attacked by a slight illness, caused by fatigue. As he had a very large diamond on his finger, and the people of the inn had taken notice of a prodigiously heavy box among his baggage, there were two physicians to attend him, though he had never sent for them, and two devotees who warmed his broths.

``I remember,'' Martin said, ``also to have been sick at Paris in my first voyage; I was very poor, thus I had neither friends, devotees, nor doctors, and I recovered.''

However, what with physic and bleeding, Candide's illness became serious. A parson of the neighborhood came with great meekness to ask for a bill for the other world payable to the bearer. Candide would do nothing for him; but the devotees assured him it was the new fashion. He answered that he was not a man of fashion. Martin wished to throw the priest out of the window. The priest swore that they would not bury Candide. Martin swore that he would bury the priest if he continued to be troublesome. The quarrel grew heated. Martin took him by the shoulders and roughly turned him out of doors; which occasioned great scandal and a law-suit.

Candide got well again, and during his convalescence he had very good company to sup with him. They played high. Candide wondered why it was that the ace never came to him; but Martin was not at all astonished.

Among those who did him the honours of the town was a little Abbe of Perigord, one of those busybodies who are ever alert, officious, forward, fawning, and complaisant; who watch for strangers in their passage through the capital, tell them the scandalous history of the town, and offer them pleasure at all prices. He first took Candide and Martin to La Comedie, where they played a new tragedy. Candide happened to be seated near some of the fashionable wits. This did not prevent his shedding tears at the well-acted scenes. One of these critics at his side said to him between the acts:

``Your tears are misplaced; that is a shocking actress; the actor who plays with her is yet worse; and the play is still worse than the actors. The author does not know a word of Arabic, yet the scene is in Arabia; moreover he is a man that does not believe in innate ideas; and I will bring you, to-morrow, twenty pamphlets written against him.''\footnote{In the 1759 editions, in place of the long passage in brackets from here to page 215, there was only the following: ``'Sir,' said the Perigordian Abbe to him, 'have you noticed that young person who has so roguish a face and so fine a figure? You may have her for ten thousand francs a month, and fifty thousand crowns in diamonds.' 'I have only a day or two to give her,' answered Candide, 'because I have a rendezvous at Venice.' In the evening after supper the insinuating Perigordian redoubled his politeness and attentions.''}

``How many dramas have you in France, sir?'' said Candide to the Abbe.

``Five or six thousand.''

``What a number!'' said Candide. ``How many good?''

``Fifteen or sixteen,'' replied the other.

``What a number!'' said Martin.

Candide was very pleased with an actress who played Queen Elizabeth in a somewhat insipid tragedy\footnote{The play referred to is supposed to be ``Le Comte d'Essex,'' by Thomas Corneille.} sometimes acted.

``That actress,'' said he to Martin, ``pleases me much; she has a likeness to Miss Cunegonde; I should be very glad to wait upon her.''

The Perigordian Abbe offered to introduce him. Candide, brought up in Germany, asked what was the etiquette, and how they treated queens of England in France.

``It is necessary to make distinctions,'' said the Abbe. ``In the provinces one takes them to the inn; in Paris, one respects them when they are beautiful, and throws them on the highway when they are dead.''\footnote{In France actors were at one time looked upon as excommunicated persons, not worthy of burial in holy ground or with Christian rites. In 1730 the ``honours of sepulture'' were refused to Mademoiselle Lecouvreur (doubtless the Miss Monime of this passage). Voltaire's miscellaneous works contain a paper on the matter.}

``Queens on the highway!'' said Candide.

``Yes, truly,'' said Martin, ``the Abbe is right. I was in Paris when Miss Monime passed, as the saying is, from this life to the other. She was refused what people call the \textit{honours of sepulture}--that is to say, of rotting with all the beggars of the neighbourhood in an ugly cemetery; she was interred all alone by her company at the corner of the Rue de Bourgogne, which ought to trouble her much, for she thought nobly.''

``That was very uncivil,'' said Candide.

``What would you have?'' said Martin; ``these people are made thus. Imagine all contradictions, all possible incompatibilities--you will find them in the government, in the law-courts, in the churches, in the public shows of this droll nation.''

``Is it true that they always laugh in Paris?'' said Candide.

``Yes,'' said the Abbe, ``but it means nothing, for they complain of everything with great fits of laughter; they even do the most detestable things while laughing.''

``Who,'' said Candide, ``is that great pig who spoke so ill of the piece at which I wept, and of the actors who gave me so much pleasure?''

``He is a bad character,'' answered the Abbe, ``who gains his livelihood by saying evil of all plays and of all books. He hates whatever succeeds, as the eunuchs hate those who enjoy; he is one of the serpents of literature who nourish themselves on dirt and spite; he is a \textit{folliculaire}.''

``What is a \textit{folliculaire}?'' said Candide.

``It is,'' said the Abbe, ``a pamphleteer--a Freron.''\footnote{Elie-Catherine Freron was a French critic (1719-1776) who incurred the enmity of Voltaire. In 1752 Freron, in \textit{Lettres sur quelques ecrits du temps}, wrote pointedly of Voltaire as one who chose to be all things to all men, and Voltaire retaliated by references such as these in \textit{Candide}.}

Thus Candide, Martin, and the Perigordian conversed on the staircase, while watching every one go out after the performance.

``Although I am eager to see Cunegonde again,'' said Candide, ``I should like to sup with Miss Clairon, for she appears to me admirable.''

The Abbe was not the man to approach Miss Clairon, who saw only good company.

``She is engaged for this evening,'' he said, ``but I shall have the honour to take you to the house of a lady of quality, and there you will know Paris as if you had lived in it for years.''

Candide, who was naturally curious, let himself be taken to this lady's house, at the end of the Faubourg St. Honore. The company was occupied in playing faro; a dozen melancholy punters held each in his hand a little pack of cards; a bad record of his misfortunes. Profound silence reigned; pallor was on the faces of the punters, anxiety on that of the banker, and the hostess, sitting near the unpitying banker, noticed with lynx-eyes all the doubled and other increased stakes, as each player dog's-eared his cards; she made them turn down the edges again with severe, but polite attention; she showed no vexation for fear of losing her customers. The lady insisted upon being called the Marchioness of Parolignac. Her daughter, aged fifteen, was among the punters, and notified with a covert glance the cheatings of the poor people who tried to repair the cruelties of fate. The Perigordian Abbe, Candide and Martin entered; no one rose, no one saluted them, no one looked at them; all were profoundly occupied with their cards.

``The Baroness of Thunder-ten-Tronckh was more polite,'' said Candide.

However, the Abbe whispered to the Marchioness, who half rose, honoured Candide with a gracious smile, and Martin with a condescending nod; she gave a seat and a pack of cards to Candide, who lost fifty thousand francs in two deals, after which they supped very gaily, and every one was astonished that Candide was not moved by his loss; the servants said among themselves, in the language of servants:--

``Some English lord is here this evening.''

The supper passed at first like most Parisian suppers, in silence, followed by a noise of words which could not be distinguished, then with pleasantries of which most were insipid, with false news, with bad reasoning, a little politics, and much evil speaking; they also discussed new books.

``Have you seen,'' said the Perigordian Abbe, ``the romance of Sieur Gauchat, doctor of divinity?''\footnote{Gabriel Gauchat (1709-1779), French ecclesiastical writer, was author of a number of works on religious subjects.}

``Yes,'' answered one of the guests, ``but I have not been able to finish it. We have a crowd of silly writings, but all together do not approach the impertinence of 'Gauchat, Doctor of Divinity.' I am so satiated with the great number of detestable books with which we are inundated that I am reduced to punting at faro.''

``And the \textit{Melanges} of Archdeacon Trublet,\footnote{Nicholas Charles Joseph Trublet (1697-1770) was a French writer whose criticism of Voltaire was revenged in passages such as this one in \textit{Candide}, and one in the \textit{Pauvre Diable} beginning: L'abbe Trublet avait alors le rage - D'etre a Paris un petit personage.} what do you say of that?'' said the Abbe.

``Ah!'' said the Marchioness of Parolignac, ``the wearisome mortal! How curiously he repeats to you all that the world knows! How heavily he discusses that which is not worth the trouble of lightly remarking upon! How, without wit, he appropriates the wit of others! How he spoils what he steals! How he disgusts me! But he will disgust me no longer--it is enough to have read a few of the Archdeacon's pages.''

There was at table a wise man of taste, who supported the Marchioness. They spoke afterwards of tragedies; the lady asked why there were tragedies which were sometimes played and which could not be read. The man of taste explained very well how a piece could have some interest, and have almost no merit; he proved in few words that it was not enough to introduce one or two of those situations which one finds in all romances, and which always seduce the spectator, but that it was necessary to be new without being odd, often sublime and always natural, to know the human heart and to make it speak; to be a great poet without allowing any person in the piece to appear to be a poet; to know language perfectly--to speak it with purity, with continuous harmony and without rhythm ever taking anything from sense.

``Whoever,'' added he, ``does not observe all these rules can produce one or two tragedies, applauded at a theatre, but he will never be counted in the ranks of good writers. There are very few good tragedies; some are idylls in dialogue, well written and well rhymed, others political reasonings which lull to sleep, or amplifications which repel; others demoniac dreams in barbarous style, interrupted in sequence, with long apostrophes to the gods, because they do not know how to speak to men, with false maxims, with bombastic commonplaces!''

Candide listened with attention to this discourse, and conceived a great idea of the speaker, and as the Marchioness had taken care to place him beside her, he leaned towards her and took the liberty of asking who was the man who had spoken so well.

``He is a scholar,'' said the lady, ``who does not play, whom the Abbe sometimes brings to supper; he is perfectly at home among tragedies and books, and he has written a tragedy which was hissed, and a book of which nothing has ever been seen outside his bookseller's shop excepting the copy which he dedicated to me.''

``The great man!'' said Candide. ``He is another Pangloss!''

Then, turning towards him, he said:

``Sir, you think doubtless that all is for the best in the moral and physical world, and that nothing could be otherwise than it is?''

``I, sir!'' answered the scholar, ``I know nothing of all that; I find that all goes awry with me; that no one knows either what is his rank, nor what is his condition, what he does nor what he ought to do; and that except supper, which is always gay, and where there appears to be enough concord, all the rest of the time is passed in impertinent quarrels; Jansenist against Molinist, Parliament against the Church, men of letters against men of letters, courtesans against courtesans, financiers against the people, wives against husbands, relatives against relatives--it is eternal war.''

``I have seen the worst,'' Candide replied. ``But a wise man, who since has had the misfortune to be hanged, taught me that all is marvellously well; these are but the shadows on a beautiful picture.''

``Your hanged man mocked the world,'' said Martin. ``The shadows are horrible blots.''

``They are men who make the blots,'' said Candide, ``and they cannot be dispensed with.''

``It is not their fault then,'' said Martin.

Most of the punters, who understood nothing of this language, drank, and Martin reasoned with the scholar, and Candide related some of his adventures to his hostess.

After supper the Marchioness took Candide into her boudoir, and made him sit upon a sofa.

``Ah, well!'' said she to him, ``you love desperately Miss Cunegonde of Thunder-ten-Tronckh?''

``Yes, madame,'' answered Candide.

The Marchioness replied to him with a tender smile:

``You answer me like a young man from Westphalia. A Frenchman would have said, 'It is true that I have loved Miss Cunegonde, but seeing you, madame, I think I no longer love her.'''

``Alas! madame,'' said Candide, ``I will answer you as you wish.''

``Your passion for her,'' said the Marchioness, ``commenced by picking up her handkerchief. I wish that you would pick up my garter.''

``With all my heart,'' said Candide. And he picked it up.

``But I wish that you would put it on,'' said the lady.

And Candide put it on.

``You see,'' said she, ``you are a foreigner. I sometimes make my Parisian lovers languish for fifteen days, but I give myself to you the first night because one must do the honours of one's country to a young man from Westphalia.''

The lady having perceived two enormous diamonds upon the hands of the young foreigner praised them with such good faith that from Candide's fingers they passed to her own.

Candide, returning with the Perigordian Abbe, felt some remorse in having been unfaithful to Miss Cunegonde. The Abbe sympathised in his trouble; he had had but a light part of the fifty thousand francs lost at play and of the value of the two brilliants, half given, half extorted. His design was to profit as much as he could by the advantages which the acquaintance of Candide could procure for him. He spoke much of Cunegonde, and Candide told him that he should ask forgiveness of that beautiful one for his infidelity when he should see her in Venice.

The Abbe redoubled his politeness and attentions, and took a tender interest in all that Candide said, in all that he did, in all that he wished to do.

``And so, sir, you have a rendezvous at Venice?''

``Yes, monsieur Abbe,'' answered Candide. ``It is absolutely necessary that I go to meet Miss Cunegonde.''

And then the pleasure of talking of that which he loved induced him to relate, according to his custom, part of his adventures with the fair Westphalian.

``I believe,'' said the Abbe, ``that Miss Cunegonde has a great deal of wit, and that she writes charming letters?''

``I have never received any from her,'' said Candide, ``for being expelled from the castle on her account I had not an opportunity for writing to her. Soon after that I heard she was dead; then I found her alive; then I lost her again; and last of all, I sent an express to her two thousand five hundred leagues from here, and I wait for an answer.''

The Abbe listened attentively, and seemed to be in a brown study. He soon took his leave of the two foreigners after a most tender embrace. The following day Candide received, on awaking, a letter couched in these terms:

``My very dear love, for eight days I have been ill in this town. I learn that you are here. I would fly to your arms if I could but move. I was informed of your passage at Bordeaux, where I left faithful Cacambo and the old woman, who are to follow me very soon. The Governor of Buenos Ayres has taken all, but there remains to me your heart. Come! your presence will either give me life or kill me with pleasure.''

This charming, this unhoped-for letter transported Candide with an inexpressible joy, and the illness of his dear Cunegonde overwhelmed him with grief. Divided between those two passions, he took his gold and his diamonds and hurried away, with Martin, to the hotel where Miss Cunegonde was lodged. He entered her room trembling, his heart palpitating, his voice sobbing; he wished to open the curtains of the bed, and asked for a light.

``Take care what you do,'' said the servant-maid; ``the light hurts her,'' and immediately she drew the curtain again.

``My dear Cunegonde,'' said Candide, weeping, ``how are you? If you cannot see me, at least speak to me.''

``She cannot speak,'' said the maid.

The lady then put a plump hand out from the bed, and Candide bathed it with his tears and afterwards filled it with diamonds, leaving a bag of gold upon the easy chair.

In the midst of these transports in came an officer, followed by the Abbe and a file of soldiers.

``There,'' said he, ``are the two suspected foreigners,'' and at the same time he ordered them to be seized and carried to prison.

``Travellers are not treated thus in El Dorado,'' said Candide.

``I am more a Manichean now than ever,'' said Martin.

``But pray, sir, where are you going to carry us?'' said Candide.

``To a dungeon,'' answered the officer.

Martin, having recovered himself a little, judged that the lady who acted the part of Cunegonde was a cheat, that the Perigordian Abbe was a knave who had imposed upon the honest simplicity of Candide, and that the officer was another knave whom they might easily silence.

Candide, advised by Martin and impatient to see the real Cunegonde, rather than expose himself before a court of justice, proposed to the officer to give him three small diamonds, each worth about three thousand pistoles.

``Ah, sir,'' said the man with the ivory baton, ``had you committed all the imaginable crimes you would be to me the most honest man in the world. Three diamonds! Each worth three thousand pistoles! Sir, instead of carrying you to jail I would lose my life to serve you. There are orders for arresting all foreigners, but leave it to me. I have a brother at Dieppe in Normandy! I'll conduct you thither, and if you have a diamond to give him he'll take as much care of you as I would.''

``And why,'' said Candide, ``should all foreigners be arrested?''

``It is,'' the Perigordian Abbe then made answer, ``because a poor beggar of the country of Atrebatie\footnote{Damiens, who attempted the life of Louis XV. in 1757, was born at Arras, capital of Artois (Atrebatie).} heard some foolish things said. This induced him to commit a parricide, not such as that of 1610 in the month of May,\footnote{On May 14, 1610, Ravaillac assassinated Henry VI.} but such as that of 1594 in the month of December,\footnote{On December 27, 1594, Jean Chatel attempted to assassinate Henry IV.} and such as others which have been committed in other years and other months by other poor devils who had heard nonsense spoken.''

The officer then explained what the Abbe meant.

``Ah, the monsters!'' cried Candide. ``What horrors among a people who dance and sing! Is there no way of getting quickly out of this country where monkeys provoke tigers? I have seen no bears in my country, but \textit{men} I have beheld nowhere except in El Dorado. In the name of God, sir, conduct me to Venice, where I am to await Miss Cunegonde.''

``I can conduct you no further than lower Normandy,'' said the officer.

Immediately he ordered his irons to be struck off, acknowledged himself mistaken, sent away his men, set out with Candide and Martin for Dieppe, and left them in the care of his brother.

There was then a small Dutch ship in the harbour. The Norman, who by the virtue of three more diamonds had become the most subservient of men, put Candide and his attendants on board a vessel that was just ready to set sail for Portsmouth in England.

This was not the way to Venice, but Candide thought he had made his way out of hell, and reckoned that he would soon have an opportunity for resuming his journey.

