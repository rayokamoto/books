\chapter{Candide's Voyage to Constantinople}
\lettrine[lraise=0.1,nindent=0em,slope=-.5em]{T}{he} faithful Cacambo had already prevailed upon the Turkish skipper, who was to conduct the Sultan Achmet to Constantinople, to receive Candide and Martin on his ship. They both embarked after having made their obeisance to his miserable Highness.

``You see,'' said Candide to Martin on the way, ``we supped with six dethroned kings, and of those six there was one to whom I gave charity. Perhaps there are many other princes yet more unfortunate. For my part, I have only lost a hundred sheep; and now I am flying into Cunegonde's arms. My dear Martin, yet once more Pangloss was right: all is for the best.''

``I wish it,'' answered Martin.

``But,'' said Candide, ``it was a very strange adventure we met with at Venice. It has never before been seen or heard that six dethroned kings have supped together at a public inn.''

``It is not more extraordinary,'' said Martin, ``than most of the things that have happened to us. It is a very common thing for kings to be dethroned; and as for the honour we have had of supping in their company, it is a trifle not worth our attention.''

No sooner had Candide got on board the vessel than he flew to his old valet and friend Cacambo, and tenderly embraced him.

``Well,'' said he, ``what news of Cunegonde? Is she still a prodigy of beauty? Does she love me still? How is she? Thou hast doubtless bought her a palace at Constantinople?''

``My dear master,'' answered Cacambo, ``Cunegonde washes dishes on the banks of the Propontis, in the service of a prince, who has very few dishes to wash; she is a slave in the family of an ancient sovereign named Ragotsky,\footnote{François Leopold Ragotsky (1676-1735).} to whom the Grand Turk allows three crowns a day in his exile. But what is worse still is, that she has lost her beauty and has become horribly ugly.''

``Well, handsome or ugly,'' replied Candide, ``I am a man of honour, and it is my duty to love her still. But how came she to be reduced to so abject a state with the five or six millions that you took to her?''

``Ah!'' said Cacambo, ``was I not to give two millions to Senor Don Fernando d'Ibaraa, y Figueora, y Mascarenes, y Lampourdos, y Souza, Governor of Buenos Ayres, for permitting Miss Cunegonde to come away? And did not a corsair bravely rob us of all the rest? Did not this corsair carry us to Cape Matapan, to Milo, to Nicaria, to Samos, to Petra, to the Dardanelles, to Marmora, to Scutari? Cunegonde and the old woman serve the prince I now mentioned to you, and I am slave to the dethroned Sultan.''

``What a series of shocking calamities!'' cried Candide. ``But after all, I have some diamonds left; and I may easily pay Cunegonde's ransom. Yet it is a pity that she is grown so ugly.''

Then, turning towards Martin: ``Who do you think,'' said he, ``is most to be pitied--the Sultan Achmet, the Emperor Ivan, King Charles Edward, or I?''

``How should I know!'' answered Martin. ``I must see into your hearts to be able to tell.''

``Ah!'' said Candide, ``if Pangloss were here, he could tell.''

``I know not,'' said Martin, ``in what sort of scales your Pangloss would weigh the misfortunes of mankind and set a just estimate on their sorrows. All that I can presume to say is, that there are millions of people upon earth who have a hundred times more to complain of than King Charles Edward, the Emperor Ivan, or the Sultan Achmet.''

``That may well be,'' said Candide.

In a few days they reached the Bosphorus, and Candide began by paying a very high ransom for Cacambo. Then without losing time, he and his companions went on board a galley, in order to search on the banks of the Propontis for his Cunegonde, however ugly she might have become.

Among the crew there were two slaves who rowed very badly, and to whose bare shoulders the Levantine captain would now and then apply blows from a bull's pizzle. Candide, from a natural impulse, looked at these two slaves more attentively than at the other oarsmen, and approached them with pity. Their features though greatly disfigured, had a slight resemblance to those of Pangloss and the unhappy Jesuit and Westphalian Baron, brother to Miss Cunegonde. This moved and saddened him. He looked at them still more attentively.

``Indeed,'' said he to Cacambo, ``if I had not seen Master Pangloss hanged, and if I had not had the misfortune to kill the Baron, I should think it was they that were rowing.''

At the names of the Baron and of Pangloss, the two galley-slaves uttered a loud cry, held fast by the seat, and let drop their oars. The captain ran up to them and redoubled his blows with the bull's pizzle.

``Stop! stop! sir,'' cried Candide. ``I will give you what money you please.''

``What! it is Candide!'' said one of the slaves.

``What! it is Candide!'' said the other.

``Do I dream?'' cried Candide; ``am I awake? or am I on board a galley? Is this the Baron whom I killed? Is this Master Pangloss whom I saw hanged?''

``It is we! it is we!'' answered they.

``Well! is this the great philosopher?'' said Martin.

``Ah! captain,'' said Candide, ``what ransom will you take for Monsieur de Thunder-ten-Tronckh, one of the first barons of the empire, and for Monsieur Pangloss, the profoundest metaphysician in Germany?''

``Dog of a Christian,'' answered the Levantine captain, ``since these two dogs of Christian slaves are barons and metaphysicians, which I doubt not are high dignities in their country, you shall give me fifty thousand sequins.''

``You shall have them, sir. Carry me back at once to Constantinople, and you shall receive the money directly. But no; carry me first to Miss Cunegonde.''

Upon the first proposal made by Candide, however, the Levantine captain had already tacked about, and made the crew ply their oars quicker than a bird cleaves the air.

Candide embraced the Baron and Pangloss a hundred times.

``And how happened it, my dear Baron, that I did not kill you? And, my dear Pangloss, how came you to life again after being hanged? And why are you both in a Turkish galley?''

``And it is true that my dear sister is in this country?'' said the Baron.

``Yes,'' answered Cacambo.

``Then I behold, once more, my dear Candide,'' cried Pangloss.

Candide presented Martin and Cacambo to them; they embraced each other, and all spoke at once. The galley flew; they were already in the port. Instantly Candide sent for a Jew, to whom he sold for fifty thousand sequins a diamond worth a hundred thousand, though the fellow swore to him by Abraham that he could give him no more. He immediately paid the ransom for the Baron and Pangloss. The latter threw himself at the feet of his deliverer, and bathed them with his tears; the former thanked him with a nod, and promised to return him the money on the first opportunity.

``But is it indeed possible that my sister can be in Turkey?'' said he.

``Nothing is more possible,'' said Cacambo, ``since she scours the dishes in the service of a Transylvanian prince.''

Candide sent directly for two Jews and sold them some more diamonds, and then they all set out together in another galley to deliver Cunegonde from slavery.

