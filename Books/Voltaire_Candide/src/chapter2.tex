\begin{center}
II\\
\textsc{What became of Candide among the Bulgarians}
\end{center}
\vspace{-0.5cm}
\rule{\textwidth}{0.5pt}
\lettrine{C}{andide, driven from terrestrial paradise}, walked a long while without knowing where, weeping, raising his eyes to heaven, turning them often towards the most magnificent of castles which imprisoned the purest of noble young ladies. He lay down to sleep without supper, in the middle of a field between two furrows. The snow fell in large flakes. Next day Candide, all benumbed, dragged himself towards the neighbouring town which was called Waldberghofftrarbk-dikdorff, having no money, dying of hunger and fatigue, he stopped sorrowfully at the door of an inn. Two men dressed in blue observed him.

``Comrade,'' said one, ``here is a well-built young fellow, and of proper height.''

They went up to Candide and very civilly invited him to dinner.

``Gentlemen,'' replied Candide, with a most engaging modesty, ``you do me great honour, but I have not wherewithal to pay my share.''

``Oh, sir,'' said one of the blues to him, ``people of your appearance and of your merit never pay anything: are you not five feet five inches high?''

``Yes, sir, that is my height,'' answered he, making a low bow.

``Come, sir, seat yourself; not only will we pay your reckoning, but we will never suffer such a man as you to want money; men are only born to assist one another.''

``You are right,'' said Candide; ``this is what I was always taught by Mr. Pangloss, and I see plainly that all is for the best.''

They begged of him to accept a few crowns. He took them, and wished to give them his note; they refused; they seated themselves at table.

``Love you not deeply?''

``Oh yes,'' answered he; ``I deeply love Miss Cunegonde.''

``No,'' said one of the gentlemen, ``we ask you if you do not deeply love the King of the Bulgarians?''

``Not at all,'' said he; ``for I have never seen him.''

``What! he is the best of kings, and we must drink his health.''

``Oh! very willingly, gentlemen,'' and he drank.

``That is enough,'' they tell him. ``Now you are the help, the support, the defender, the hero of the Bulgarians. Your fortune is made, and your glory is assured.''

Instantly they fettered him, and carried him away to the regiment. There he was made to wheel about to the right, and to the left, to draw his rammer, to return his rammer, to present, to fire, to march, and they gave him thirty blows with a cudgel. The next day he did his exercise a little less badly, and he received but twenty blows. The day following they gave him only ten, and he was regarded by his comrades as a prodigy.

Candide, all stupefied, could not yet very well realise how he was a hero. He resolved one fine day in spring to go for a walk, marching straight before him, believing that it was a privilege of the human as well as of the animal species to make use of their legs as they pleased. He had advanced two leagues when he was overtaken by four others, heroes of six feet, who bound him and carried him to a dungeon. He was asked which he would like the best, to be whipped six-and-thirty times through all the regiment, or to receive at once twelve balls of lead in his brain. He vainly said that human will is free, and that he chose neither the one nor the other. He was forced to make a choice; he determined, in virtue of that gift of God called liberty, to run the gauntlet six-and-thirty times. He bore this twice. The regiment was composed of two thousand men; that composed for him four thousand strokes, which laid bare all his muscles and nerves, from the nape of his neck quite down to his rump. As they were going to proceed to a third whipping, Candide, able to bear no more, begged as a favour that they would be so good as to shoot him. He obtained this favour; they bandaged his eyes, and bade him kneel down. The King of the Bulgarians passed at this moment and ascertained the nature of the crime. As he had great talent, he understood from all that he learnt of Candide that he was a young metaphysician, extremely ignorant of the things of this world, and he accorded him his pardon with a clemency which will bring him praise in all the journals, and throughout all ages.

An able surgeon cured Candide in three weeks by means of emollients taught by Dioscorides. He had already a little skin, and was able to march when the King of the Bulgarians gave battle to the King of the Abares.\footnote{The Abares were a tribe of Tartars settled on the shores of the Danube, who later dwelt in part of Circassia.}
