\begin{center}
XII\\
\textsc{The Adventures of the Old Woman continued}
\end{center}
\vspace{-0.5cm}
\rule{\textwidth}{0.5pt}
\lettrine{``A}{stonished and delighted to hear my} native language, and no less surprised at what this man said, I made answer that there were much greater misfortunes than that of which he complained. I told him in a few words of the horrors which I had endured, and fainted a second time. He carried me to a neighbouring house, put me to bed, gave me food, waited upon me, consoled me, flattered me; he told me that he had never seen any one so beautiful as I, and that he never so much regretted the loss of what it was impossible to recover.

``I was born at Naples,'' said he, ``there they geld two or three thousand children every year; some die of the operation, others acquire a voice more beautiful than that of women, and others are raised to offices of state.\footnote{Carlo Broschi, called Farinelli, an Italian singer, born at Naples in 1705, without being exactly Minister, governed Spain under Ferdinand VI.; he died in 1782. He has been made one of the chief persons in one of the comic operas of MM. Auber and Scribe.} This operation was performed on me with great success and I was chapel musician to madam, the Princess of Palestrina.''

``To my mother!'' cried I.

``Your mother!'' cried he, weeping. ``What! can you be that young princess whom I brought up until the age of six years, and who promised so early to be as beautiful as you?''

``It is I, indeed; but my mother lies four hundred yards hence, torn in quarters, under a heap of dead bodies.''

``I told him all my adventures, and he made me acquainted with his; telling me that he had been sent to the Emperor of Morocco by a Christian power, to conclude a treaty with that prince, in consequence of which he was to be furnished with military stores and ships to help to demolish the commerce of other Christian Governments.

``'My mission is done,' said this honest eunuch; 'I go to embark for Ceuta, and will take you to Italy. \textit{Ma che sciagura d'essere senza coglioni!}'

``I thanked him with tears of commiseration; and instead of taking me to Italy he conducted me to Algiers, where he sold me to the Dey. Scarcely was I sold, than the plague which had made the tour of Africa, Asia, and Europe, broke out with great malignancy in Algiers. You have seen earthquakes; but pray, miss, have you ever had the plague?''

``Never,'' answered Cunegonde.

``If you had,'' said the old woman, ``you would acknowledge that it is far more terrible than an earthquake. It is common in Africa, and I caught it. Imagine to yourself the distressed situation of the daughter of a Pope, only fifteen years old, who, in less than three months, had felt the miseries of poverty and slavery, had been ravished almost every day, had beheld her mother drawn in quarters, had experienced famine and war, and was dying of the plague in Algiers. I did not die, however, but my eunuch, and the Dey, and almost the whole seraglio of Algiers perished.

``As soon as the first fury of this terrible pestilence was over, a sale was made of the Dey's slaves; I was purchased by a merchant, and carried to Tunis; this man sold me to another merchant, who sold me again to another at Tripoli; from Tripoli I was sold to Alexandria, from Alexandria to Smyrna, and from Smyrna to Constantinople. At length I became the property of an Aga of the Janissaries, who was soon ordered away to the defence of Azof, then besieged by the Russians.

``The Aga, who was a very gallant man, took his whole seraglio with him, and lodged us in a small fort on the Palus Meotides, guarded by two black eunuchs and twenty soldiers. The Turks killed prodigious numbers of the Russians, but the latter had their revenge. Azof was destroyed by fire, the inhabitants put to the sword, neither sex nor age was spared; until there remained only our little fort, and the enemy wanted to starve us out. The twenty Janissaries had sworn they would never surrender. The extremities of famine to which they were reduced, obliged them to eat our two eunuchs, for fear of violating their oath. And at the end of a few days they resolved also to devour the women.

``We had a very pious and humane Iman, who preached an excellent sermon, exhorting them not to kill us all at once.

``'Only cut off a buttock of each of those ladies,' said he, 'and you'll fare extremely well; if you must go to it again, there will be the same entertainment a few days hence; heaven will accept of so charitable an action, and send you relief.'

``He had great eloquence; he persuaded them; we underwent this terrible operation. The Iman applied the same balsam to us, as he does to children after circumcision; and we all nearly died.

``Scarcely had the Janissaries finished the repast with which we had furnished them, than the Russians came in flat-bottomed boats; not a Janissary escaped. The Russians paid no attention to the condition we were in. There are French surgeons in all parts of the world; one of them who was very clever took us under his care--he cured us; and as long as I live I shall remember that as soon as my wounds were healed he made proposals to me. He bid us all be of good cheer, telling us that the like had happened in many sieges, and that it was according to the laws of war.

``As soon as my companions could walk, they were obliged to set out for Moscow. I fell to the share of a Boyard who made me his gardener, and gave me twenty lashes a day. But this nobleman having in two years' time been broke upon the wheel along with thirty more Boyards for some broils at court, I profited by that event; I fled. I traversed all Russia; I was a long time an inn-holder's servant at Riga, the same at Rostock, at Vismar, at Leipzig, at Cassel, at Utrecht, at Leyden, at the Hague, at Rotterdam. I waxed old in misery and disgrace, having only one-half of my posteriors, and always remembering I was a Pope's daughter. A hundred times I was upon the point of killing myself; but still I loved life. This ridiculous foible is perhaps one of our most fatal characteristics; for is there anything more absurd than to wish to carry continually a burden which one can always throw down? to detest existence and yet to cling to one's existence? in brief, to caress the serpent which devours us, till he has eaten our very heart?

``In the different countries which it has been my lot to traverse, and the numerous inns where I have been servant, I have taken notice of a vast number of people who held their own existence in abhorrence, and yet I never knew of more than eight who voluntarily put an end to their misery; three negroes, four Englishmen, and a German professor named Robek.\footnote{Jean Robeck, a Swede, who was born in 1672, will be found mentioned in Rousseau's \textit{Nouvelle Heloise}. He drowned himself in the Weser at Bremen in 1729, and was the author of a Latin treatise on voluntary death, first printed in 1735.} I ended by being servant to the Jew, Don Issachar, who placed me near your presence, my fair lady. I am determined to share your fate, and have been much more affected with your misfortunes than with my own. I would never even have spoken to you of my misfortunes, had you not piqued me a little, and if it were not customary to tell stories on board a ship in order to pass away the time. In short, Miss Cunegonde, I have had experience, I know the world; therefore I advise you to divert yourself, and prevail upon each passenger to tell his story; and if there be one of them all, that has not cursed his life many a time, that has not frequently looked upon himself as the unhappiest of mortals, I give you leave to throw me headforemost into the sea.''

