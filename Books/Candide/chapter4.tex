\begin{center}
IV\\
\textsc{How Cadide found his old master Pangloss, and what happened to them}
\end{center}
\vspace{-0.5cm}
\rule{\textwidth}{0.5pt}
\lettrine{C}{andide, yet more moved with} compassion than with horror, gave to this shocking beggar the two florins which he had received from the honest Anabaptist James. The spectre looked at him very earnestly, dropped a few tears, and fell upon his neck. Candide recoiled in disgust.

``Alas!'' said one wretch to the other, ``do you no longer know your dear Pangloss?''

``What do I hear? You, my dear master! you in this terrible plight! What misfortune has happened to you? Why are you no longer in the most magnificent of castles? What has become of Miss Cunegonde, the pearl of girls, and nature's masterpiece?''

``I am so weak that I cannot stand,'' said Pangloss.

Upon which Candide carried him to the Anabaptist's stable, and gave him a crust of bread. As soon as Pangloss had refreshed himself a little:

``Well,'' said Candide, ``Cunegonde?''

``She is dead,'' replied the other.

Candide fainted at this word; his friend recalled his senses with a little bad vinegar which he found by chance in the stable. Candide reopened his eyes.

``Cunegonde is dead! Ah, best of worlds, where art thou? But of what illness did she die? Was it not for grief, upon seeing her father kick me out of his magnificent castle?''

``No,'' said Pangloss, ``she was ripped open by the Bulgarian soldiers, after having been violated by many; they broke the Baron's head for attempting to defend her; my lady, her mother, was cut in pieces; my poor pupil was served just in the same manner as his sister; and as for the castle, they have not left one stone upon another, not a barn, nor a sheep, nor a duck, nor a tree; but we have had our revenge, for the Abares have done the very same thing to a neighbouring barony, which belonged to a Bulgarian lord.''

At this discourse Candide fainted again; but coming to himself, and having said all that it became him to say, inquired into the cause and effect, as well as into the \textit{sufficient reason} that had reduced Pangloss to so miserable a plight.

``Alas!'' said the other, ``it was love; love, the comfort of the human species, the preserver of the universe, the soul of all sensible beings, love, tender love.''

``Alas!'' said Candide, ``I know this love, that sovereign of hearts, that soul of our souls; yet it never cost me more than a kiss and twenty kicks on the backside. How could this beautiful cause produce in you an effect so abominable?''

Pangloss made answer in these terms: ``Oh, my dear Candide, you remember Paquette, that pretty wench who waited on our noble Baroness; in her arms I tasted the delights of paradise, which produced in me those hell torments with which you see me devoured; she was infected with them, she is perhaps dead of them. This present Paquette received of a learned Grey Friar, who had traced it to its source; he had had it of an old countess, who had received it from a cavalry captain, who owed it to a marchioness, who took it from a page, who had received it from a Jesuit, who when a novice had it in a direct line from one of the companions of Christopher Columbus.\footnote{Venereal disease was said to have been first brought from Hispaniola, in the West Indies, by some followers of Columbus who were later employed in the siege of Naples. From this latter circumstance it was at one time known as the Neapolitan disease.} For my part I shall give it to nobody, I am dying.''

``Oh, Pangloss!'' cried Candide, ``what a strange genealogy! Is not the Devil the original stock of it?''

``Not at all,'' replied this great man, ``it was a thing unavoidable, a necessary ingredient in the best of worlds; for if Columbus had not in an island of America caught this disease, which contaminates the source of life, frequently even hinders generation, and which is evidently opposed to the great end of nature, we should have neither chocolate nor cochineal. We are also to observe that upon our continent, this distemper is like religious controversy, confined to a particular spot. The Turks, the Indians, the Persians, the Chinese, the Siamese, the Japanese, know nothing of it; but there is a sufficient reason for believing that they will know it in their turn in a few centuries. In the meantime, it has made marvellous progress among us, especially in those great armies composed of honest well-disciplined hirelings, who decide the destiny of states; for we may safely affirm that when an army of thirty thousand men fights another of an equal number, there are about twenty thousand of them p-x-d on each side.''

``Well, this is wonderful!'' said Candide, ``but you must get cured.''

``Alas! how can I?'' said Pangloss, ``I have not a farthing, my friend, and all over the globe there is no letting of blood or taking a glister, without paying, or somebody paying for you.''

These last words determined Candide; he went and flung himself at the feet of the charitable Anabaptist James, and gave him so touching a picture of the state to which his friend was reduced, that the good man did not scruple to take Dr. Pangloss into his house, and had him cured at his expense. In the cure Pangloss lost only an eye and an ear. He wrote well, and knew arithmetic perfectly. The Anabaptist James made him his bookkeeper. At the end of two months, being obliged to go to Lisbon about some mercantile affairs, he took the two philosophers with him in his ship. Pangloss explained to him how everything was so constituted that it could not be better. James was not of this opinion.

``It is more likely,'' said he, ``mankind have a little corrupted nature, for they were not born wolves, and they have become wolves; God has given them neither cannon of four-and-twenty pounders, nor bayonets; and yet they have made cannon and bayonets to destroy one another. Into this account I might throw not only bankrupts, but Justice which seizes on the effects of bankrupts to cheat the creditors.''

``All this was indispensable,'' replied the one-eyed doctor, ``for private misfortunes make the general good, so that the more private misfortunes there are the greater is the general good.''

While he reasoned, the sky darkened, the winds blew from the four quarters, and the ship was assailed by a most terrible tempest within sight of the port of Lisbon.

