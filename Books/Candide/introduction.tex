\chapter*{Introduction}
\addcontentsline{toc}{chapter}{Introduction to Candide}
Ever since 1759, when Voltaire wrote ``Candide'' in ridicule of the notion
that this is the best of all possible worlds, this world has been a
gayer place for readers. Voltaire wrote it in three days, and five or
six generations have found that its laughter does not grow old.

``Candide'' has not aged. Yet how different the book would have looked if
Voltaire had written it a hundred and fifty years later than 1759. It
would have been, among other things, a book of sights and sounds. A
modern writer would have tried to catch and fix in words some of those
Atlantic changes which broke the Atlantic monotony of that voyage from
Cadiz to Buenos Ayres. When Martin and Candide were sailing the length
of the Mediterranean we should have had a contrast between naked scarped
Balearic cliffs and headlands of Calabria in their mists. We should have
had quarter distances, far horizons, the altering silhouettes of an
Ionian island. Colored birds would have filled Paraguay with their
silver or acid cries.

Dr. Pangloss, to prove the existence of design in the universe, says
that noses were made to carry spectacles, and so we have spectacles. A
modern satirist would not try to paint with Voltaire's quick brush the
doctrine that he wanted to expose. And he would choose a more
complicated doctrine than Dr. Pangloss's optimism, would study it more
closely, feel his destructive way about it with a more learned and
caressing malice. His attack, stealthier, more flexible and more patient
than Voltaire's, would call upon us, especially when his learning got a
little out of control, to be more than patient. Now and then he would
bore us. ``Candide'' never bored anybody except William Wordsworth.

Voltaire's men and women point his case against optimism by starting
high and falling low. A modern could not go about it after this fashion.
He would not plunge his people into an unfamiliar misery. He would just
keep them in the misery they were born to.

But such an account of Voltaire's procedure is as misleading as the
plaster cast of a dance. Look at his procedure again. Mademoiselle
Cunégonde, the illustrious Westphalian, sprung from a family that could
prove seventy-one quarterings, descends and descends until we find her
earning her keep by washing dishes in the Propontis. The aged faithful
attendant, victim of a hundred acts of rape by negro pirates, remembers
that she is the daughter of a pope, and that in honor of her
approaching marriage with a Prince of Massa-Carrara all Italy wrote
sonnets of which not one was passable. We do not need to know French
literature before Voltaire in order to feel, although the lurking parody
may escape us, that he is poking fun at us and at himself. His laughter
at his own methods grows more unmistakable at the last, when he
caricatures them by casually assembling six fallen monarchs in an inn at
Venice.

A modern assailant of optimism would arm himself with social pity. There
is no social pity in ``Candide.'' Voltaire, whose light touch on familiar
institutions opens them and reveals their absurdity, likes to remind us
that the slaughter and pillage and murder which Candide witnessed among
the Bulgarians was perfectly regular, having been conducted according to
the laws and usages of war. Had Voltaire lived today he would have done
to poverty what he did to war. Pitying the poor, he would have shown us
poverty as a ridiculous anachronism, and both the ridicule and the pity
would have expressed his indignation.

Almost any modern, essaying a philosophic tale, would make it long.
``Candide'' is only a ``Hamlet'' and a half long. It would hardly have been
shorter if Voltaire had spent three months on it, instead of those three
days. A conciseness to be matched in English by nobody except Pope, who
can say a plagiarizing enemy "steals much, spends little, and has
nothing left," a conciseness which Pope toiled and sweated for, came as
easy as wit to Voltaire. He can afford to be witty, parenthetically, by
the way, prodigally, without saving, because he knows there is more wit
where that came from.

One of Max Beerbohm's cartoons shows us the young Twentieth Century
going at top speed, and watched by two of his predecessors. Underneath
is this legend: "The Grave Misgivings of the Nineteenth Century, and the
Wicked Amusement of the Eighteenth, in Watching the Progress (or
whatever it is) of the Twentieth." This Eighteenth Century snuff-taking
and malicious, is like Voltaire, who nevertheless must know, if he
happens to think of it, that not yet in the Twentieth Century, not for
all its speed mania, has any one come near to equalling the speed of a
prose tale by Voltaire. ``Candide'' is a full book. It is filled with
mockery, with inventiveness, with things as concrete as things to eat
and coins, it has time for the neatest intellectual clickings, it is
never hurried, and it moves with the most amazing rapidity. It has the
rapidity of high spirits playing a game. The dry high spirits of this
destroyer of optimism make most optimists look damp and depressed.
Contemplation of the stupidity which deems happiness possible almost
made Voltaire happy. His attack on optimism is one of the gayest books
in the world. Gaiety has been scattered everywhere up and down its pages
by Voltaire's lavish hand, by his thin fingers.

Many propagandist satirical books have been written with ``Candide'' in
mind, but not too many. Today, especially, when new faiths are changing
the structure of the world, faiths which are still plastic enough to be
deformed by every disciple, each disciple for himself, and which have
not yet received the final deformation known as universal acceptance,
today ``Candide'' is an inspiration to every narrative satirist who hates
one of these new faiths, or hates every interpretation of it but his
own. Either hatred will serve as a motive to satire.

That is why the present is one of the right moments to republish
``Candide.'' I hope it will inspire younger men and women, the only ones
who can be inspired, to have a try at Theodore, or Militarism; Jane, or
Pacifism; at So-and-So, the Pragmatist or the Freudian. And I hope, too,
that they will without trying hold their pens with an eighteenth century
lightness, not inappropriate to a philosophic tale. In Voltaire's
fingers, as Anatole France has said, the pen runs and laughs. \\
\rightline{\textsc{Philip Littell}}
