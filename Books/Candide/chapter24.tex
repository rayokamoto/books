\begin{center}
XXIV\\
\textsc{Of Paquette and Friar Giroflée}
\end{center}
\vspace{-0.5cm}
\rule{\textwidth}{0.5pt}
\lettrine{U}{pon their arrival at Venice}, Candide went to search for Cacambo at every inn and coffee-house, and among all the ladies of pleasure, but to no purpose. He sent every day to inquire on all the ships that came in. But there was no news of Cacambo.

``What!'' said he to Martin, ``I have had time to voyage from Surinam to Bordeaux, to go from Bordeaux to Paris, from Paris to Dieppe, from Dieppe to Portsmouth, to coast along Portugal and Spain, to cross the whole Mediterranean, to spend some months, and yet the beautiful Cunegonde has not arrived! Instead of her I have only met a Parisian wench and a Perigordian Abbe. Cunegonde is dead without doubt, and there is nothing for me but to die. Alas! how much better it would have been for me to have remained in the paradise of El Dorado than to come back to this cursed Europe! You are in the right, my dear Martin: all is misery and illusion.''

He fell into a deep melancholy, and neither went to see the opera, nor any of the other diversions of the Carnival; nay, he was proof against the temptations of all the ladies.

``You are in truth very simple,'' said Martin to him, ``if you imagine that a mongrel valet, who has five or six millions in his pocket, will go to the other end of the world to seek your mistress and bring her to you to Venice. If he find her, he will keep her to himself; if he do not find her he will get another. I advise you to forget your valet Cacambo and your mistress Cunegonde.''

Martin was not consoling. Candide's melancholy increased; and Martin continued to prove to him that there was very little virtue or happiness upon earth, except perhaps in El Dorado, where nobody could gain admittance.

While they were disputing on this important subject and waiting for Cunegonde, Candide saw a young Theatin friar in St. Mark's Piazza, holding a girl on his arm. The Theatin looked fresh coloured, plump, and vigorous; his eyes were sparkling, his air assured, his look lofty, and his step bold. The girl was very pretty, and sang; she looked amorously at her Theatin, and from time to time pinched his fat cheeks.

``At least you will allow me,'' said Candide to Martin, ``that these two are happy. Hitherto I have met with none but unfortunate people in the whole habitable globe, except in El Dorado; but as to this pair, I would venture to lay a wager that they are very happy.''

``I lay you they are not,'' said Martin.

``We need only ask them to dine with us,'' said Candide, ``and you will see whether I am mistaken.''

Immediately he accosted them, presented his compliments, and invited them to his inn to eat some macaroni, with Lombard partridges, and caviare, and to drink some Montepulciano, Lachrymae Christi, Cyprus and Samos wine. The girl blushed, the Theatin accepted the invitation and she followed him, casting her eyes on Candide with confusion and surprise, and dropping a few tears. No sooner had she set foot in Candide's apartment than she cried out:

``Ah! Mr. Candide does not know Paquette again.''

Candide had not viewed her as yet with attention, his thoughts being entirely taken up with Cunegonde; but recollecting her as she spoke.

``Alas!'' said he, ``my poor child, it is you who reduced Doctor Pangloss to the beautiful condition in which I saw him?''

``Alas! it was I, sir, indeed,'' answered Paquette. ``I see that you have heard all. I have been informed of the frightful disasters that befell the family of my lady Baroness, and the fair Cunegonde. I swear to you that my fate has been scarcely less sad. I was very innocent when you knew me. A Grey Friar, who was my confessor, easily seduced me. The consequences were terrible. I was obliged to quit the castle some time after the Baron had sent you away with kicks on the backside. If a famous surgeon had not taken compassion on me, I should have died. For some time I was this surgeon's mistress, merely out of gratitude. His wife, who was mad with jealousy, beat me every day unmercifully; she was a fury. The surgeon was one of the ugliest of men, and I the most wretched of women, to be continually beaten for a man I did not love. You know, sir, what a dangerous thing it is for an ill-natured woman to be married to a doctor. Incensed at the behaviour of his wife, he one day gave her so effectual a remedy to cure her of a slight cold, that she died two hours after, in most horrid convulsions. The wife's relations prosecuted the husband; he took flight, and I was thrown into jail. My innocence would not have saved me if I had not been good-looking. The judge set me free, on condition that he succeeded the surgeon. I was soon supplanted by a rival, turned out of doors quite destitute, and obliged to continue this abominable trade, which appears so pleasant to you men, while to us women it is the utmost abyss of misery. I have come to exercise the profession at Venice. Ah! sir, if you could only imagine what it is to be obliged to caress indifferently an old merchant, a lawyer, a monk, a gondolier, an abbe, to be exposed to abuse and insults; to be often reduced to borrowing a petticoat, only to go and have it raised by a disagreeable man; to be robbed by one of what one has earned from another; to be subject to the extortions of the officers of justice; and to have in prospect only a frightful old age, a hospital, and a dung-hill; you would conclude that I am one of the most unhappy creatures in the world.''\footnote{Commenting upon this passage, M. Sarcey says admirably: ``All is there! In those ten lines Voltaire has gathered all the griefs and all the terrors of these creatures; the picture is admirable for its truth and power! But do you not feel the pity and sympathy of the painter? Here irony becomes sad, and in a way an avenger. Voltaire cries out with horror against the society which throws some of its members into such an abyss. He has his `Bartholomew' fever; we tremble with him through contagion.''}

Paquette thus opened her heart to honest Candide, in the presence of Martin, who said to his friend:

``You see that already I have won half the wager.''

Friar Giroflee stayed in the dining-room, and drank a glass or two of wine while he was waiting for dinner.

``But,'' said Candide to Paquette, ``you looked so gay and content when I met you; you sang and you behaved so lovingly to the Theatin, that you seemed to me as happy as you pretend to be now the reverse.''

``Ah! sir,'' answered Paquette, ``this is one of the miseries of the trade. Yesterday I was robbed and beaten by an officer; yet to-day I must put on good humour to please a friar.''

Candide wanted no more convincing; he owned that Martin was in the right. They sat down to table with Paquette and the Theatin; the repast was entertaining; and towards the end they conversed with all confidence.

``Father,'' said Candide to the Friar, ``you appear to me to enjoy a state that all the world might envy; the flower of health shines in your face, your expression makes plain your happiness; you have a very pretty girl for your recreation, and you seem well satisfied with your state as a Theatin.''

``My faith, sir,'' said Friar Giroflee, ``I wish that all the Theatins were at the bottom of the sea. I have been tempted a hundred times to set fire to the convent, and go and become a Turk. My parents forced me at the age of fifteen to put on this detestable habit, to increase the fortune of a cursed elder brother, whom God confound. Jealousy, discord, and fury, dwell in the convent. It is true I have preached a few bad sermons that have brought me in a little money, of which the prior stole half, while the rest serves to maintain my girls; but when I return at night to the monastery, I am ready to dash my head against the walls of the dormitory; and all my fellows are in the same case.''

Martin turned towards Candide with his usual coolness.

``Well,'' said he, ``have I not won the whole wager?''

Candide gave two thousand piastres to Paquette, and one thousand to Friar Giroflee.

``I'll answer for it,'' said he, ``that with this they will be happy.''

``I do not believe it at all,'' said Martin; ``you will, perhaps, with these piastres only render them the more unhappy.''

``Let that be as it may,'' said Candide, ``but one thing consoles me. I see that we often meet with those whom we expected never to see more; so that, perhaps, as I have found my red sheep and Paquette, it may well be that I shall also find Cunegonde.''

``I wish,'' said Martin, ``she may one day make you very happy; but I doubt it very much.''

``You are very hard of belief,'' said Candide.

``I have lived,'' said Martin.

``You see those gondoliers,'' said Candide, ``are they not perpetually singing?''

``You do not see them,'' said Martin, ``at home with their wives and brats. The Doge has his troubles, the gondoliers have theirs. It is true that, all things considered, the life of a gondolier is preferable to that of a Doge; but I believe the difference to be so trifling that it is not worth the trouble of examining.''

``People talk,'' said Candide, ``of the Senator Pococurante, who lives in that fine palace on the Brenta, where he entertains foreigners in the politest manner. They pretend that this man has never felt any uneasiness.''

``I should be glad to see such a rarity,'' said Martin.

Candide immediately sent to ask the Lord Pococurante permission to wait upon him the next day.

