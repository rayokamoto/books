\vspace{1cm}
\begingroup
\let\clearpage\relax
\chapter{The Conclusion}
\thispagestyle{pter}
\endgroup
\vspace{-1cm}
\lettrine[lraise=0.1,nindent=0em,slope=-.5em]{A}{t} the bottom of his heart Candide had no wish to marry Cunegonde. But the extreme impertinence of the Baron determined him to conclude the match, and Cunegonde pressed him so strongly that he could not go from his word. He consulted Pangloss, Martin, and the faithful Cacambo. Pangloss drew up an excellent memorial, wherein he proved that the Baron had no right over his sister, and that according to all the laws of the empire, she might marry Candide with her left hand. Martin was for throwing the Baron into the sea; Cacambo decided that it would be better to deliver him up again to the captain of the galley, after which they thought to send him back to the General Father of the Order at Rome by the first ship. This advice was well received, the old woman approved it; they said not a word to his sister; the thing was executed for a little money, and they had the double pleasure of entrapping a Jesuit, and punishing the pride of a German baron.

It is natural to imagine that after so many disasters Candide married, and living with the philosopher Pangloss, the philosopher Martin, the prudent Cacambo, and the old woman, having besides brought so many diamonds from the country of the ancient Incas, must have led a very happy life. But he was so much imposed upon by the Jews that he had nothing left except his small farm; his wife became uglier every day, more peevish and unsupportable; the old woman was infirm and even more fretful than Cunegonde. Cacambo, who worked in the garden, and took vegetables for sale to Constantinople, was fatigued with hard work, and cursed his destiny. Pangloss was in despair at not shining in some German university. For Martin, he was firmly persuaded that he would be as badly off elsewhere, and therefore bore things patiently. Candide, Martin, and Pangloss sometimes disputed about morals and metaphysics. They often saw passing under the windows of their farm boats full of Effendis, Pashas, and Cadis, who were going into banishment to Lemnos, Mitylene, or Erzeroum. And they saw other Cadis, Pashas, and Effendis coming to supply the place of the exiles, and afterwards exiled in their turn. They saw heads decently impaled for presentation to the Sublime Porte. Such spectacles as these increased the number of their dissertations; and when they did not dispute time hung so heavily upon their hands, that one day the old woman ventured to say to them:

``I want to know which is worse, to be ravished a hundred times by negro pirates, to have a buttock cut off, to run the gauntlet among the Bulgarians, to be whipped and hanged at an \textit{auto-da-fe}, to be dissected, to row in the galleys--in short, to go through all the miseries we have undergone, or to stay here and have nothing to do?''

``It is a great question,'' said Candide.

This discourse gave rise to new reflections, and Martin especially concluded that man was born to live either in a state of distracting inquietude or of lethargic disgust. Candide did not quite agree to that, but he affirmed nothing. Pangloss owned that he had always suffered horribly, but as he had once asserted that everything went wonderfully well, he asserted it still, though he no longer believed it.

What helped to confirm Martin in his detestable principles, to stagger Candide more than ever, and to puzzle Pangloss, was that one day they saw Paquette and Friar Giroflee land at the farm in extreme misery. They had soon squandered their three thousand piastres, parted, were reconciled, quarrelled again, were thrown into gaol, had escaped, and Friar Giroflee had at length become Turk. Paquette continued her trade wherever she went, but made nothing of it.

``I foresaw,'' said Martin to Candide, ``that your presents would soon be dissipated, and only make them the more miserable. You have rolled in millions of money, you and Cacambo; and yet you are not happier than Friar Giroflee and Paquette.''

``Ha!'' said Pangloss to Paquette, ``Providence has then brought you amongst us again, my poor child! Do you know that you cost me the tip of my nose, an eye, and an ear, as you may see? What a world is this!''

And now this new adventure set them philosophising more than ever.

In the neighbourhood there lived a very famous Dervish who was esteemed the best philosopher in all Turkey, and they went to consult him. Pangloss was the speaker.

``Master,'' said he, ``we come to beg you to tell why so strange an animal as man was made.''

``With what meddlest thou?'' said the Dervish; ``is it thy business?''

``But, reverend father,'' said Candide, ``there is horrible evil in this world.''

``What signifies it,'' said the Dervish, ``whether there be evil or good? When his highness sends a ship to Egypt, does he trouble his head whether the mice on board are at their ease or not?''

``What, then, must we do?'' said Pangloss.

``Hold your tongue,'' answered the Dervish.

``I was in hopes,'' said Pangloss, ``that I should reason with you a little about causes and effects, about the best of possible worlds, the origin of evil, the nature of the soul, and the pre-established harmony.''

At these words, the Dervish shut the door in their faces.

During this conversation, the news was spread that two Viziers and the Mufti had been strangled at Constantinople, and that several of their friends had been impaled. This catastrophe made a great noise for some hours. Pangloss, Candide, and Martin, returning to the little farm, saw a good old man taking the fresh air at his door under an orange bower. Pangloss, who was as inquisitive as he was argumentative, asked the old man what was the name of the strangled Mufti.

``I do not know,'' answered the worthy man, ``and I have not known the name of any Mufti, nor of any Vizier. I am entirely ignorant of the event you mention; I presume in general that they who meddle with the administration of public affairs die sometimes miserably, and that they deserve it; but I never trouble my head about what is transacting at Constantinople; I content myself with sending there for sale the fruits of the garden which I cultivate.''

Having said these words, he invited the strangers into his house; his two sons and two daughters presented them with several sorts of sherbet, which they made themselves, with Kaimak enriched with the candied-peel of citrons, with oranges, lemons, pine-apples, pistachio-nuts, and Mocha coffee unadulterated with the bad coffee of Batavia or the American islands. After which the two daughters of the honest Mussulman perfumed the strangers' beards.

``You must have a vast and magnificent estate,'' said Candide to the Turk.

``I have only twenty acres,'' replied the old man; ``I and my children cultivate them; our labour preserves us from three great evils--weariness, vice, and want.''

Candide, on his way home, made profound reflections on the old man's conversation.

``This honest Turk,'' said he to Pangloss and Martin, ``seems to be in a situation far preferable to that of the six kings with whom we had the honour of supping.''

``Grandeur,'' said Pangloss, ``is extremely dangerous according to the testimony of philosophers. For, in short, Eglon, King of Moab, was assassinated by Ehud; Absalom was hung by his hair, and pierced with three darts; King Nadab, the son of Jeroboam, was killed by Baasa; King Ela by Zimri; Ahaziah by Jehu; Athaliah by Jehoiada; the Kings Jehoiakim, Jeconiah, and Zedekiah, were led into captivity. You know how perished Croesus, Astyages, Darius, Dionysius of Syracuse, Pyrrhus, Perseus, Hannibal, Jugurtha, Ariovistus, Caesar, Pompey, Nero, Otho, Vitellius, Domitian, Richard II. of England, Edward II., Henry VI., Richard III., Mary Stuart, Charles I., the three Henrys of France, the Emperor Henry IV.! You know----''

``I know also,'' said Candide, ``that we must cultivate our garden.''

``You are right,'' said Pangloss, ``for when man was first placed in the Garden of Eden, he was put there \textit{ut operaretur eum}, that he might cultivate it; which shows that man was not born to be idle.''

``Let us work,'' said Martin, ``without disputing; it is the only way to render life tolerable.''

The whole little society entered into this laudable design, according to their different abilities. Their little plot of land produced plentiful crops. Cunegonde was, indeed, very ugly, but she became an excellent pastry cook; Paquette worked at embroidery; the old woman looked after the linen. They were all, not excepting Friar Giroflee, of some service or other; for he made a good joiner, and became a very honest man.

Pangloss sometimes said to Candide:

``There is a concatenation of events in this best of all possible worlds: for if you had not been kicked out of a magnificent castle for love of Miss Cunegonde: if you had not been put into the Inquisition: if you had not walked over America: if you had not stabbed the Baron: if you had not lost all your sheep from the fine country of El Dorado: you would not be here eating preserved citrons and pistachio-nuts.''

``All that is very well,'' answered Candide, ``but let us cultivate our garden.''

\begin{center}
\vspace{1cm}
\textsc{Fin}	
\end{center}
