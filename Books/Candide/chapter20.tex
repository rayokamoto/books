\begin{center}
XX\\
\textsc{What happened at Sea to Candide and Martin}
\end{center}
\vspace{-0.5cm}
\rule{\textwidth}{0.5pt}
\lettrine{T}{he old philosopher, whose name} was Martin, embarked then with Candide for Bordeaux. They had both seen and suffered a great deal; and if the vessel had sailed from Surinam to Japan, by the Cape of Good Hope, the subject of moral and natural evil would have enabled them to entertain one another during the whole voyage.

Candide, however, had one great advantage over Martin, in that he always hoped to see Miss Cunegonde; whereas Martin had nothing at all to hope. Besides, Candide was possessed of money and jewels, and though he had lost one hundred large red sheep, laden with the greatest treasure upon earth; though the knavery of the Dutch skipper still sat heavy upon his mind; yet when he reflected upon what he had still left, and when he mentioned the name of Cunegonde, especially towards the latter end of a repast, he inclined to Pangloss's doctrine.

``But you, Mr. Martin,'' said he to the philosopher, ``what do you think of all this? what are your ideas on moral and natural evil?''

``Sir,'' answered Martin, ``our priests accused me of being a Socinian, but the real fact is I am a Manichean.''\footnote{\textit{Socinians}; followers of the teaching of Lalius and Faustus Socinus (16th century), which denied the doctrine of the Trinity, the deity of Christ, the personality of the devil, the native and total depravity of man, the vicarious atonement and eternal punishment. The Socinians are now represented by the Unitarians. \textit{Manicheans}; followers of Manes or Manichaeus (3rd century), a Persian who maintained that there are two principles, the one good and the other evil, each equally powerful in the government of the world.}

``You jest,'' said Candide; ``there are no longer Manicheans in the world.''

``I am one,'' said Martin. ``I cannot help it; I know not how to think otherwise.''

``Surely you must be possessed by the devil,'' said Candide.

``He is so deeply concerned in the affairs of this world,'' answered Martin, ``that he may very well be in me, as well as in everybody else; but I own to you that when I cast an eye on this globe, or rather on this little ball, I cannot help thinking that God has abandoned it to some malignant being. I except, always, El Dorado. I scarcely ever knew a city that did not desire the destruction of a neighbouring city, nor a family that did not wish to exterminate some other family. Everywhere the weak execrate the powerful, before whom they cringe; and the powerful beat them like sheep whose wool and flesh they sell. A million regimented assassins, from one extremity of Europe to the other, get their bread by disciplined depredation and murder, for want of more honest employment. Even in those cities which seem to enjoy peace, and where the arts flourish, the inhabitants are devoured by more envy, care, and uneasiness than are experienced by a besieged town. Secret griefs are more cruel than public calamities. In a word I have seen so much, and experienced so much that I am a Manichean.''

``There are, however, some things good,'' said Candide.

``That may be,'' said Martin; ``but I know them not.''

In the middle of this dispute they heard the report of cannon; it redoubled every instant. Each took out his glass. They saw two ships in close fight about three miles off. The wind brought both so near to the French vessel that our travellers had the pleasure of seeing the fight at their ease. At length one let off a broadside, so low and so truly aimed, that the other sank to the bottom. Candide and Martin could plainly perceive a hundred men on the deck of the sinking vessel; they raised their hands to heaven and uttered terrible outcries, and the next moment were swallowed up by the sea.

``Well,'' said Martin, ``this is how men treat one another.''

``It is true,'' said Candide; ``there is something diabolical in this affair.''

While speaking, he saw he knew not what, of a shining red, swimming close to the vessel. They put out the long-boat to see what it could be: it was one of his sheep! Candide was more rejoiced at the recovery of this one sheep than he had been grieved at the loss of the hundred laden with the large diamonds of El Dorado.

The French captain soon saw that the captain of the victorious vessel was a Spaniard, and that the other was a Dutch pirate, and the very same one who had robbed Candide. The immense plunder which this villain had amassed, was buried with him in the sea, and out of the whole only one sheep was saved.

``You see,'' said Candide to Martin, ``that crime is sometimes punished. This rogue of a Dutch skipper has met with the fate he deserved.''

``Yes,'' said Martin; ``but why should the passengers be doomed also to destruction? God has punished the knave, and the devil has drowned the rest.''

The French and Spanish ships continued their course, and Candide continued his conversation with Martin. They disputed fifteen successive days, and on the last of those fifteen days, they were as far advanced as on the first. But, however, they chatted, they communicated ideas, they consoled each other. Candide caressed his sheep.

``Since I have found thee again,'' said he, ``I may likewise chance to find my Cunegonde.''

