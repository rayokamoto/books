\chapter{How Candide found Cunegonde and the Old Woman again}
\thispagestyle{pter}
\vspace{-1cm}While Candide, the Baron, Pangloss, Martin, and Cacambo were relating their several adventures, were reasoning on the contingent or non-contingent events of the universe, disputing on effects and causes, on moral and physical evil, on liberty and necessity, and on the consolations a slave may feel even on a Turkish galley, they arrived at the house of the Transylvanian prince on the banks of the Propontis. The first objects which met their sight were Cunegonde and the old woman hanging towels out to dry.

The Baron paled at this sight. The tender, loving Candide, seeing his beautiful Cunegonde embrowned, with blood-shot eyes, withered neck, wrinkled cheeks, and rough, red arms, recoiled three paces, seized with horror, and then advanced out of good manners. She embraced Candide and her brother; they embraced the old woman, and Candide ransomed them both.

There was a small farm in the neighbourhood which the old woman proposed to Candide to make a shift with till the company could be provided for in a better manner. Cunegonde did not know she had grown ugly, for nobody had told her of it; and she reminded Candide of his promise in so positive a tone that the good man durst not refuse her. He therefore intimated to the Baron that he intended marrying his sister.

``I will not suffer,'' said the Baron, ``such meanness on her part, and such insolence on yours; I will never be reproached with this scandalous thing; my sister's children would never be able to enter the church in Germany. No; my sister shall only marry a baron of the empire.''

Cunegonde flung herself at his feet, and bathed them with her tears; still he was inflexible.

``Thou foolish fellow,'' said Candide; ``I have delivered thee out of the galleys, I have paid thy ransom, and thy sister's also; she was a scullion, and is very ugly, yet I am so condescending as to marry her; and dost thou pretend to oppose the match? I should kill thee again, were I only to consult my anger.''

``Thou mayest kill me again,'' said the Baron, ``but thou shalt not marry my sister, at least whilst I am living.''

