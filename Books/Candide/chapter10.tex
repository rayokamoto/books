\vspace{1cm}
\begingroup
\let\clearpage\relax
\chapter{In what distress Candide, Cunegonde, and the Old Woman arrived at Cadiz; and of their Embarkation}
\thispagestyle{pter}
\endgroup
\vspace{-1cm}
\lettrine[lraise=0.1,nindent=0em,slope=-.5em]{``W}{ho} was it that robbed me of my money and jewels?'' said Cunegonde, all bathed in tears. ``How shall we live? What shall we do? Where find Inquisitors or Jews who will give me more?''

``Alas!'' said the old woman, ``I have a shrewd suspicion of a reverend Grey Friar, who stayed last night in the same inn with us at Badajos. God preserve me from judging rashly, but he came into our room twice, and he set out upon his journey long before us.''

``Alas!'' said Candide, ``dear Pangloss has often demonstrated to me that the goods of this world are common to all men, and that each has an equal right to them. But according to these principles the Grey Friar ought to have left us enough to carry us through our journey. Have you nothing at all left, my dear Cunegonde?''

``Not a farthing,'' said she.

``What then must we do?'' said Candide.

``Sell one of the horses,'' replied the old woman. ``I will ride behind Miss Cunegonde, though I can hold myself only on one buttock, and we shall reach Cadiz.''

In the same inn there was a Benedictine prior who bought the horse for a cheap price. Candide, Cunegonde, and the old woman, having passed through Lucena, Chillas, and Lebrixa, arrived at length at Cadiz. A fleet was there getting ready, and troops assembling to bring to reason the reverend Jesuit Fathers of Paraguay, accused of having made one of the native tribes in the neighborhood of San Sacrament revolt against the Kings of Spain and Portugal. Candide having been in the Bulgarian service, performed the military exercise before the general of this little army with so graceful an address, with so intrepid an air, and with such agility and expedition, that he was given the command of a company of foot. Now, he was a captain! He set sail with Miss Cunegonde, the old woman, two valets, and the two Andalusian horses, which had belonged to the grand Inquisitor of Portugal.

During their voyage they reasoned a good deal on the philosophy of poor Pangloss.

``We are going into another world,'' said Candide; ``and surely it must be there that all is for the best. For I must confess there is reason to complain a little of what passeth in our world in regard to both natural and moral philosophy.''

``I love you with all my heart,'' said Cunegonde; ``but my soul is still full of fright at that which I have seen and experienced.''

``All will be well,'' replied Candide; ``the sea of this new world is already better than our European sea; it is calmer, the winds more regular. It is certainly the New World which is the best of all possible worlds.''

``God grant it,'' said Cunegonde; ``but I have been so horribly unhappy there that my heart is almost closed to hope.''

``You complain,'' said the old woman; ``alas! you have not known such misfortunes as mine.''

Cunegonde almost broke out laughing, finding the good woman very amusing, for pretending to have been as unfortunate as she.

``Alas!'' said Cunegonde, ``my good mother, unless you have been ravished by two Bulgarians, have received two deep wounds in your belly, have had two castles demolished, have had two mothers cut to pieces before your eyes, and two of your lovers whipped at an \textit{auto-da-fe}, I do not conceive how you could be more unfortunate than I. Add that I was born a baroness of seventy-two quarterings--and have been a cook!''

``Miss,'' replied the old woman, ``you do not know my birth; and were I to show you my backside, you would not talk in that manner, but would suspend your judgment.''

This speech having raised extreme curiosity in the minds of Cunegonde and Candide, the old woman spoke to them as follows.

